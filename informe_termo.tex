\documentclass[]{article}
\usepackage[spanish]{babel} 
\usepackage{amsmath} 
\usepackage[colorlinks=true]{hyperref}
\usepackage{enumitem} 
\usepackage{graphicx}   
\usepackage[a4paper,top=3cm,bottom=3cm,left=3cm,right=3cm,marginparwidth=1.75cm]{geometry} 
%\graphicspath{ {./img/} } 

\title{Informe I}
\author{}
\date{\today}
%==================================================================================
\begin{document}
\maketitle  
"Hola mundo"

\tableofcontents
%========Introducción
%\section{Introdución}
Esto es una prueba.



%========Objetivos
%\section{Objetivos}
%\begin{itemize}
    %\item Comprobar usando la simulación, las leyes de gases ideales.
    %\item Obtener modelos gráficos y matemáticos que relacionen las magnitudes termodinámicas presión,
    %volumen y temperatura.
%\end{itemize}


%=========Marco Teórico
%\section{Marco Teórico}




%=========Materiales
%\section{Materiales}


%=========Procedimiento
\section{Procedimiento}
\begin{enumerate}
    \item Para la primera simulación, se trabajará a una temperatura constante de 300K y un número de partículas pesadas n = 50. Luego se  variará el ancho del recipiente (15nm, 13nm, 11nm,
    9nm, 7nm y 5nm). Después se registrá cada valor de presión obtenido de la simulación, para cada
    uno de los anchos respectivos.
    Una vez terminada la primera recolección de datos, se repetirá la simulación para partículas pesadas y ligeras, ocupando temperaturas constantes de 300K y 600K, y n = 50, n = 100, n = 150
    para cada caso. 

    \item Para la segunda simulación, se trabajará con un número de partículas pesadas n = 50, y una
    temperatura inicial de 300K. La presión en este caso estará oscilando entre 5,4atm y 6,3atm
    aproximadamente, debemos hacerla constante en alguno de estos valores, donde el recipiente inicial mide 10nm. Luego, 
    variamos la temperatura con el regulador y debemos fijarnos que ocurre en las siguientes temperaturas: 
    150K, 225K, 375K y 450K, registrando los datos de las variaciones del ancho del recipiente.
    

    \item Repetiremos la simulación anterior con un  n = 150, una temperatura inicial de 300K, el recipiente
    tendrá un ancho inicial de 10nm, la presión estará oscilando entre 17,1atm y 17,9atm, aproximadamente, la haremos constante
    en alguno de estos valores. Volvemos a variar la temperatura entre 150K, 225K, 375K y 450K. Registre los datos de las variaciones 
    del ancho del recipiente en cada una de las temperaturas dadas.
    Grafique la relación eligiendo los datos de temperatura y volumen adecuados para los ejes x (temperatura) e y (volumen) para que 
    su función modele los datos lo más próximo posible al experimento.

    \item  Por último, volvemos a realizar la simulación con un n = 250, una temperatura inicial de 300K, el
    recipiente tendra un ancho inicial de 10nm, la presión estará oscilando entre 28,8atm y 29,6atm,
    aproximadamente, la haremos constante en alguno de estos valores. Volvemos a variar la temperatura entre 
    150K, 225K, 375K y 450K.
    
    
\end{enumerate}










%========Resultados
\section{Resultados}

 




\begin{table}
 \centering
 \begin{tabular}{|c|c|c|l|} \hline
  n (mol)&  T = 300K        &    T = 600K    &        \\ \hline
   50    & [3.4 - 4.4], 15 & [7.3 - 8.2], 15& P atm, L nm\\ 
         & [4.0 - 5.0], 13 & [8.6 - 9.5], 13& P atm, L nm\\
         & [4.8 - 5.8], 11 & [10.2 - 11.1], 11& P atm, L nm\\ 
         & [6.0 - 7.0],  9 & [12.5 - 13.4],  9& P atm, L nm\\ 
         & [7.9 - 8.8],  7 & [16.2 - 17.1],  7& P atm, L nm\\  
         & [11.2 - 12.1], 5 & [22.9 - 23.8],  5& P atm, L nm\\ \hline
        %%%%%%%%%%%%%%%%%%%%%%%%%%%%%%%%%%%%%%%%%%%%%%%%%%%%%
   100   & [7.3 - 8.2], 15 & [15.1 - 16.0], 15& P atm, L nm\\ 
         & [8.5 - 9.4], 13 & [17.5 - 18.4], 13& P atm, L nm\\
         & [10.2 - 11.1], 11 & [20.9 - 21.7], 11& P atm, L nm\\ 
         & [12.5 - 13.4], 9 & [25.6 - 26.4],  9& P atm, L nm\\ 
         & [16.3 - 17.2], 7 & [32.9 - 33.7],  7& P atm, L nm\\  
         & [22.9 - 23.8] , 5 & [46.3 - 47.1],  5& P atm, L nm\\ \hline
         %%%%%%%%%%%%%%%%%%%%%%%%%%%%%%%%%%%%%%%%%%%%%%%%%%%%%
   150   & [11.2 - 12.1], 15 & [22.9 - 23.8], 15& P atm, L nm\\ 
         & [13.1 - 14.0], 13 & [26.4 - 27.3], 13& P atm, L nm\\
         & [15.5 - 16.3], 11 & [31.4 - 32.2], 11& P atm, L nm\\ 
         & [19.1 - 19.9],  9 & [38.7 - 39.5],  9& P atm, L nm\\ 
         & [24.7 - 25.5],  7 & [49.4 - 50.1],  7& P atm, L nm\\  
         & [34.6 - 35.3],  5 & [69.8 - 70.4],  5& P atm, L nm\\ \hline

   

 \end{tabular}
 \caption{\label{tab:Pesadas} Datos para partículas pesadas.}
 \end{table}



\begin{table}
\centering
\begin{tabular}{|c|c|c|l|} \hline
    n (mol) & P atm & P atm &       \\ \hline
    50      &       &       & T Kelvin, L nm \\ \hline
    100     &       &       & T Kelvin, L nm \\ \hline
    150     &       &       & T Kelvin, L nm \\ \hline

\end{tabular}
\caption{\label{tab:V.Ancho} Variación del ancho, con respecto a la temperatura, manteniendo P cte.}    
\end{table}

















































%========Analisis
%\section{Análisis}


%========Conclusión
%\section{Conclusión}

























\end{document}
