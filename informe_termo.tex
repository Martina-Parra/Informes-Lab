\documentclass[]{article}
\usepackage[spanish]{babel} 
\usepackage{amsmath} 
\usepackage[colorlinks=true]{hyperref}
\usepackage{enumitem} 
\usepackage{graphicx}   
\usepackage[a4paper,top=3cm,bottom=3cm,left=3cm,right=3cm,marginparwidth=1.75cm]{geometry} 
%\graphicspath{ {./img/} } 

%\title{Informe I}
%\author{}
%\date{\today}
%==================================================================================
\begin{document}
\begin{titlepage}
      \centering                   %Para poner logo udec 
      %{\includegraphics[width=0.2\textwidth]{imágenes\logoudec}\par}  %{nombre carpeta\nombreimagen}
      %\vspace{1cm}
      {\bfseries\LARGE Universidad de concepcion \par}
      \vspace{1cm}
      {\scshape\Large Facultad de ciencias fisícas y matematicas \par}
      \vspace{2cm}
      {\scshape\Huge Laboratorio 1 \par}
      \vspace{2cm}
      {\itshape\Large Proyecto laboratorio termodinamica \par}
      \vfill
      {\Large Autores: \par}
      {\Large Martina Contreras, Noemí De la peña, Benjamín Opazo. \par}
      \vfill
      \vfill
      {\Large Profesor: \par}
      {\Large Claudio Alonso Faúndez Araya \par}
      \vfill
      \vfill
      {\Large Carrera: \par}
      {\Large Ciencias fisícas \par}
      \vfill
      \vfill
      {\Large Ayudantes: \par}
      {\Large Arelly Nunez y Anahis Verana \par}
      \vfill
      {\Large Septiembre 2022 \par}
      \end{titlepage}
%\maketitle  

\tableofcontents

%========Introducción
\section{Introdución}
En este informe presentaremos series de datos obtenidos en la simulación de laboratorio, en la cual se realizo por medio de un simulador para medir distintas propiedades termodinámicas de gases ideales.
En el cual primero definiremos que es un gas ideal y como se relacionan las propiedades termodinámicas p, V y T. Para luego definir los materiales que usaremos en nuestro laboratorio.
Además, exhibiremos tablas de datos donde se hizo cariar tanto la longitud del recipiente como la temperatura, donde estas serán representadas en gráficos V-P y T-V de los cuales obtendremos información con la que podremos responder las preguntas propuestas y obtener conclusiones.
Los objetivos del laboratorio son:
 \begin{itemize}  %Objetivos
      \item Comprobar usando la simulación, las leyes de gases ideales.
      \item Obtener modelos gráficos y matemáticos que relacionen las magnitudes termodinámicas presión,
      volumen y temperatura.
\end{itemize}


%=========Marco Teórico
\section{Marco Teórico}
Gas ideal: Modelo idealizado que representa muy bien el comportamiento 
de los gases en algunas circunstancias (Como a presiones bajas)[1]
Las caracteristicas de un gas ideal son:
\begin{itemize}
      \item Las particulas del gas no tienen volumen (Ocupan el volumen del envase que los contiene)
      \item No se tienen en cuenta las interacciones de atracción y repulsión molecular.
\end{itemize}
Las propiedades termodinámicas de un gas ideal tienen las siguientes relaciones entre ellos.
\begin{equation*}
      PV=nRT
\end{equation*}
donde :
P= Presión[atm]   , V= Volumen total [m^3] , R= Constante de los gases  y  T= Temperatura [K°]
%=========Materiales

\section{Materiales}

\begin{itemize}
      \item Recipiente con gas.
      \item Pistón
      \item Terómetro.
      \item Barómetro.
      \item Regularizador de temperatura.
      \item Bomba de moleculas.
\end{itemize}

%=========Procedimiento
\section{Procedimiento}
\begin{enumerate}
    \item Para la primera simulación, se trabajará a una temperatura constante de 300K y un número de partículas pesadas n = 50. Luego se  variará el ancho del recipiente (15nm, 13nm, 11nm,
    9nm, 7nm y 5nm). Después se registrá cada valor de presión obtenido de la simulación, para cada
    uno de los anchos respectivos.
    Una vez terminada la primera recolección de datos, se repetirá la simulación para partículas pesadas y ligeras, ocupando temperaturas constantes de 300K y 600K, y n = 50, n = 100, n = 150
    para cada caso. 

    \item Para la segunda simulación, se trabajará con un número de partículas pesadas n = 50, y una
    temperatura inicial de 300K. La presión en este caso estará oscilando entre 5,4atm y 6,3atm
    aproximadamente, debemos hacerla constante en alguno de estos valores, donde el recipiente inicial mide 10nm. Luego, 
    variamos la temperatura con el regulador y debemos fijarnos que ocurre en las siguientes temperaturas: 
    150K, 225K, 375K y 450K, registrando los datos de las variaciones del ancho del recipiente.
    

    \item Repetiremos la simulación anterior con un  n = 150, una temperatura inicial de 300K, el recipiente
    tendrá un ancho inicial de 10nm, la presión estará oscilando entre 17,1atm y 17,9atm, aproximadamente, la haremos constante
    en alguno de estos valores. Volvemos a variar la temperatura entre 150K, 225K, 375K y 450K. Registre los datos de las variaciones 
    del ancho del recipiente en cada una de las temperaturas dadas.
    Grafique la relación eligiendo los datos de temperatura y volumen adecuados para los ejes x (temperatura) e y (volumen) para que 
    su función modele los datos lo más próximo posible al experimento.

    \item  Por último, volvemos a realizar la simulación con un n = 250, una temperatura inicial de 300K, el
    recipiente tendra un ancho inicial de 10nm, la presión estará oscilando entre 28,8atm y 29,6atm,
    aproximadamente, la haremos constante en alguno de estos valores. Volvemos a variar la temperatura entre 
    150K, 225K, 375K y 450K.
    
    
\end{enumerate}









%========Resultados
\section{Resultados}

 




\begin{table}[h]
 \centering
 \begin{tabular}{|c|c|c|l|} \hline
  n (mol)&  T = 300K        &    T = 600K    &        \\ \hline
   50    & [3.4 - 4.4], 15 & [7.3 - 8.2], 15& P atm, L nm\\ 
         & [4.0 - 5.0], 13 & [8.6 - 9.5], 13& P atm, L nm\\
         & [4.8 - 5.8], 11 & [10.2 - 11.1], 11& P atm, L nm\\ 
         & [6.0 - 7.0],  9 & [12.5 - 13.4],  9& P atm, L nm\\ 
         & [7.9 - 8.8],  7 & [16.2 - 17.1],  7& P atm, L nm\\  
         & [11.2 - 12.1], 5 & [22.9 - 23.8],  5& P atm, L nm\\ \hline
        %%%%%%%%%%%%%%%%%%%%%%%%%%%%%%%%%%%%%%%%%%%%%%%%%%%%%
   100   & [7.3 - 8.2], 15 & [15.1 - 16.0], 15& P atm, L nm\\ 
         & [8.5 - 9.4], 13 & [17.5 - 18.4], 13& P atm, L nm\\
         & [10.2 - 11.1], 11 & [20.9 - 21.7], 11& P atm, L nm\\ 
         & [12.5 - 13.4], 9 & [25.6 - 26.4],  9& P atm, L nm\\ 
         & [16.3 - 17.2], 7 & [32.9 - 33.7],  7& P atm, L nm\\  
         & [22.9 - 23.8] , 5 & [46.3 - 47.1],  5& P atm, L nm\\ \hline
         %%%%%%%%%%%%%%%%%%%%%%%%%%%%%%%%%%%%%%%%%%%%%%%%%%%%%
   150   & [11.2 - 12.1], 15 & [22.9 - 23.8], 15& P atm, L nm\\ 
         & [13.1 - 14.0], 13 & [26.4 - 27.3], 13& P atm, L nm\\
         & [15.5 - 16.3], 11 & [31.4 - 32.2], 11& P atm, L nm\\ 
         & [19.1 - 19.9],  9 & [38.7 - 39.5],  9& P atm, L nm\\ 
         & [24.7 - 25.5],  7 & [49.4 - 50.1],  7& P atm, L nm\\  
         & [34.6 - 35.3],  5 & [69.8 - 70.4],  5& P atm, L nm\\ \hline

   

\end{tabular}
\caption{\label{tab:Pesadas} Datos para partículas pesadas.}
\end{table}

\begin{table}[h]
 \centering
\begin{tabular}{|c|c|c|l|} \hline
  n (mol)&  T = 300K        &    T = 600K    &        \\ \hline
   50    & [3.7 - 4.9], 15 & [7.6 - 8.1], 15& P atm, L nm\\ 
         & [4.1 - 4.9], 13 & [8.6 - 9.4], 13& P atm, L nm\\
         & [4.9 - 5.8], 11 & [10.2 - 11.0], 11& P atm, L nm\\ 
         & [6.2 - 6.9],  9 & [12.7 - 13.3],  9& P atm, L nm\\ 
         & [7.8 - 8.5],  7 & [16.6 - 17.0],  7& P atm, L nm\\  
         & [11.0 - 12.1], 5 & [23.0 - 23.7],  5& P atm, L nm\\ \hline
        %%%%%%%%%%%%%%%%%%%%%%%%%%%%%%%%%%%%%%%%%%%%%%%%%%%%%
   100   & [7.6 - 8.2], 15 & [15.2 - 15.9], 15& P atm, L nm\\ 
         & [8.5 - 9.4], 13 & [17.6 - 18.3], 13& P atm, L nm\\
         & [10.4 - 11.0], 11 & [20.9 - 21.6], 11& P atm, L nm\\ 
         & [12.6 - 13.1], 9 & [25.5 - 26.3],  9& P atm, L nm\\ 
         & [16.5 - 17.1], 7 & [33.1 - 33.9],  7& P atm, L nm\\  
         & [22.9 - 23.7] , 5 & [46.3 - 47.0],  5& P atm, L nm\\ \hline
         %%%%%%%%%%%%%%%%%%%%%%%%%%%%%%%%%%%%%%%%%%%%%%%%%%%%%
   150   & [11.5 - 12.1], 15 & [23.0 - 23.7], 15& P atm, L nm\\ 
         & [13.1 - 13.6], 13 & [26.5 - 27.3], 13& P atm, L nm\\
         & [15.7 - 16.2], 11 & [31.5 - 32.2], 11& P atm, L nm\\ 
         & [19.1 - 19.7],  9 & [38.5 - 39.1],  9& P atm, L nm\\ 
         & [24.9 - 25.6],  7 & [49.4 - 50.5],  7& P atm, L nm\\  
         & [34.7 - 35.3],  5 & [69.8 - 70.3],  5& P atm, L nm\\ \hline

\end{tabular}
\caption{\label{tab:ligeras} Datos para partículas ligeras.}
\end{table}

 


\begin{table}
 \centering
 \begin{tabular}{|c|c|l|} \hline
  n (mol)&      P atm      &    T Kelvin, L nm\\ \hline
   50    &                 &             150, 5.0\\
         &  5.8            &             225, 7.5\\
         &                 &             375, 12.5\\
         &                 &             450, 15\\ \hline
            %%%%%%%%%%%%%%%%%%%%%%%%%%%%%%%%%%%%%%%%
   100   &                 &             150, 5.0\\ 
         &  17.5           &             225, 7.5\\
         &                 &             375, 12.5\\ 
         &                 &             450, 15\\ \hline                
            %%%%%%%%%%%%%%%%%%%%%%%%%%%%%%%%%%%%%%%%%
   150   &                 &              150, 5.0\\ 
         &  29.2           &              225, 7.5\\
         &                 &              375, 12.5\\ 
         &                 &              450, 15\\ \hline          

 \end{tabular}
 \caption{\label{tab:V.Ancho} Variación del ancho, con respecto a la temperatura, manteniendo P cte.}    
 \end{table}








%========Analisis
%\section{Análisis}


%========Conclusión
%\section{Conclusión}

























\end{document}
