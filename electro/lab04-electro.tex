\documentclass[]{article}
\usepackage[spanish]{babel} 
\usepackage{amsmath} 
\usepackage[colorlinks=true]{hyperref}
\usepackage{enumitem} 
\usepackage{graphicx}   
\usepackage[a4paper,top=2.5cm,bottom=2.5cm,left=2cm,right=2.5cm]{geometry} 
\usepackage[]{subfigure}
\usepackage[]{multicol}
\setlength{\columnsep}{1cm}
\usepackage[]{hyperref}
%\usepackage[maxbibnames=99, sorting=none]{biblatex}
\usepackage{amssymb}
\usepackage[]{txfonts}

%\usepackage[authoryear]{natbib}
\newenvironment{Figura}
{\par\medskip\noindent\minipage{\linewidth}}
{\endminipage\par\medskip}



%==================================================================================
\begin{document}
\begin{titlepage}
      \begin{center}     
              
            \includegraphics[width=0.2\textwidth]{lab03/img/escudo_udec.png}                       %Para poner logo udec   %{nombre carpeta\nombreimagen}
            
            
            
            \vspace{1cm}
            \textsc{{\LARGE Universidad de Concepción}}
            
            \vspace{1cm}
            {\scshape\Large Facultad de ciencias fisícas y matemáticas \par}
            \vspace{2cm}
            {\scshape\Huge Laboratorio 4 \par}
            \vspace{2cm}
            {\itshape\Large Proyecto laboratorio termodinámica \par}
            \vfill
            {\Large Autores: \par}
            {\Large Martina Contreras, Noemí De La Peña, Benjamín Opazo. \par}
            \vfill
            \vfill
            {\Large Profesor: \par}
            {\Large Juan Pablo Staforelli \par}
            \vfill
            \vfill
            {\Large Carrera: \par}
            {\Large Ciencias fisícas \par}
            \vfill
            \vfill
            {\Large Ayudante: \par}
            {\Large Fernanda Paz Vera \par}
            \vfill
            {\Large Noviembre 2022 \par}
      \end{center}
\end{titlepage}            
 

\tableofcontents
\newpage



%==================================================================================



% \begin{abstract}

% \end{abstract}



%========Introducción=============
\section{Introdución}

\begin{itemize}
      \item Conocer el funcionamiento de un osciloscopio.
      \item Concocer el funcionamiento de un generador de señales alternas.
      \item Aprender a medir voltaje y tiempo con osciloscopio.
      \item Comprobar experimentalmente las ecuaciones de carga y descarga de condensadores. \textit{Circuito RL}
      \item Comprobar experimentalmente las ecuaciones de conexión y desconexión de bobinas de autoinducción. \textit{Circuito RC}
      \item Obtener oscilaciones de carga y medir la mfrecuencia propia del sistema oscilante. \textit{Circuito RLC}
\end{itemize}



%==============Marco teorico============
\section{Marco teórico}
\begin{enumerate}
      \item Características de ondas de voltaje y corriente alternas sinusoidales. \\
      Conmsideremos las siguientes definiciones: 
      \begin{itemize}
            \item $v(t) = V_m sin(wt)$: valor instantáneo de voltaje alterno sinusoidal,
            \item $i(t) = I_m sin(wt)$: valor instantáneo de corriente alterna sinusoidal,
            \item $V_m$: amplitud o valor máximo de la onda de voltaje,
            \item $I_m$: amplitud o valor máximo de la onda de corriente,
            \item $w = 2 \pi f$,
            \item $f = \frac{1}{T}$: frecuencia de oscilación,
            \item $T$: período de la alternancia.
      \end{itemize}
      Para comparar valores de voltaje y corriente continuos con los correspondientes alternos se definen los llamados valores efectivos.
      El valor efectivo de un voltaje o una corriente alterna, resulta de igualar la energía eléctrica de corriente continua con la energía eléctrica de corriente alterna, 
      cuando ambas se transforman en calor Q en una misma resistencia R. Si la comparación se efectúa durante un lapso de tiempo igual a un periodo T, se tendrá la igualdad:
      \begin{align*}
            Q_{cc} (R,T) = Q_{cA}(R,T)
      \end{align*}
      Cantidades que se pueden expresar en términos del voltaje aplicado sobre R o de la corriente que circula por R.
       Así, para el primer caso, se tiene:
      \begin{align*}
            \frac{V^2}{R}T = \int_0 ^T \frac{v^2(t)}{R}dt \Rightarrow V = \left(\frac{1}{T} \int_0 ^T V_m ^2 sin^2 (wt) dt \right)^{1/2}
      \end{align*}
      de donde $V = \frac{1}{2} \sqrt{2}V_m$: voltaje efectivo y para la corriente:
      \begin{align*}
            I^2 RT = \int_0 ^T i^2 (t)R dt \Rightarrow I = \left(\frac{1}{T} \int_0 ^T I_m ^2 sin^2 (wt) dt \right)^{1/2}
      \end{align*}
      De donde $I = \frac{1}{2} \sqrt{2}I_m$: corriente efectiva. \\
      Los instrumentos medidores de voltaje y corriente alternos están calibrados para medir valores efectivos.
      
      \item Circuitos RC, RL, RLC en condición transitoria.
      \begin{enumerate}
            \item \textbf{Circuito RC} \\
             Para el circuito de la figura \ref{fig: circuito-RC} y a partir del instante $(t = 0)$ de conexión de S en 1, estando C
             completamente cargado $(q = 0)$, la ecuación diferencial y su solución para la carga $q(t)$ se escriben
             \begin{align*}
                  \frac{dq}{dt} + \frac{q}{RC} = \frac{V_0}{R} \Rightarrow q(t) = V_0 C(1-e^{-\frac{t}{RC}})
             \end{align*}
            de donde resultan los voltajes $v_C(t)$ en el condensador y $v_R (t)$ en el resistor, siguientes:
            \begin{align*}
                  v_C(t) = \frac{q(t)}{C} = V_0 (1-e^{-\frac{t}{RC}})
            \end{align*}
            \begin{align*}
                  v_R(t) = \frac{i(t)}{R} = R \frac{dq}{dt} = V_0 (1-e^{-\frac{t}{RC}})
            \end{align*}
            Cuando $C$ ha alcanzado el estado estacionario $(t \rightarrow \infty )$, su carga ha llegado al valor
            $Q = V_0 C$. Se cambia ahora $S$ a la posición 2 para efectuar la descarga de $C$. Al quedar excluida la fuente
            $V_0$, lam ecuación diferencial y su solución, resultan ser:
            \begin{align*}
                  \frac{dq}{dt} + \frac{q}{RC} = 0 \Rightarrow q(t) = V_0 C(1-e^{-\frac{t}{RC}})
             \end{align*}
             entonces:
             \begin{align*}
                  v_C(t) =  V_0 e^{-\frac{t}{RC}}
            \end{align*}
            \begin{align*}
                  v_R(t) = - V_0 e^{-\frac{t}{RC}}
            \end{align*}
            
            \item \textbf{Circuito RL}\\
            En el circuito de la figura \ref{fig: circuito-RL}, al cerrar $S$ en $1$ se obtiene la ecuación diferencial para la conexión de $L$.
            La ecuación y su solución resultan: 
            \begin{align*}
                  \frac{di}{dt} + \frac{R}{L}i = \frac{V_0}{L} \Rightarrow i(t) = \frac{V_0}{R}(1-e^{\frac{-R}{L}t})
            \end{align*}
            desde donde:

            \begin{align*}
                  v_L(t) = L \frac{di}{dt} = V_0 e^{-\frac{R}{L}t}
            \end{align*}
            \begin{align*}
                  v_R(t) = i(t)R =  V_0 (1 - e^{-\frac{R}{L}t})
            \end{align*}
            
            Al cambiar el interruptor $S$ de $1$ a $2$, se inicia la etapa de desconexión de $L$, con ecuación diferencial y solución dadas por:
            \begin{align*}
                  \frac{di}{dt} + \frac{Ri}{L} = 0 \Rightarrow i(t) = \frac{V_0}{R} e^{\frac{R}{L}t} 
            \end{align*}
            entonces:
            \begin{align*}
                  v_L(t) = L \frac{di}{dt} = - V_0 e^{-\frac{R}{L}t}
            \end{align*}
            \begin{align*}
                  v_R(t) = i(t)R =  V_0 e^{-\frac{R}{L}t}
            \end{align*}

            \item \textbf{Circuito RLC}
            En el circuito de la figura \ref{fig: circuito-RLC} $(a)$, el condensador $C$ se encuentra cargado con carga $Q$ y la energía total
            $U_C = \frac{Q^2}{2C}$. Al cerrar el interruptor $S$. el condensador iniciará la descarga y con todo ello un proceso de 
            intercamcio de energía entre $C$ y $L$. La energía total distribuida en todo momento en $L$ y en $C$, irá decreciendo debido a 
            las pérdidas por efecto de calor disipado en $R$, todo lo cual se expresa por la ecuación:
            \begin{align*}
                  \frac{d}{dt}\left(\frac{q^2}{2C} + \frac{Li^2}{2}\right) = - i^2 R
            \end{align*} 

            que resulta en la ecuación diferencial:
            \begin{align*}
                  L \frac{d^2}{dt^2}q + R\frac{dq}{dt} + \frac{q}{C} = 0
            \end{align*}
            y su solución oscilatoria:

            \begin{align*}
                  q(t) = Q^{-\frac{R}{2L}t}cos(w't + \phi),
            \end{align*}
            con

            \begin{align*}
                  w' = .\left(w^2 - \left(\frac{R}{2L}\right)^2 \right)^{1/2}
            \end{align*}

            y $w = \frac{1}{\sqrt{LC}}$, así el voltaje instantáneo en el condensador, $V_c (t)$, tiene la expresión:
            \begin{align*}
                  V_c(t) = \frac{q(t)}{C} = \frac{Q^{-\frac{R}{2L}t}cos(w't + \phi)}{C}
            \end{align*}

            que corresponde a una oscilación amortiguada exponencialmente.
      \end{enumerate}

\end{enumerate}

\begin{figure}
      \centering
      \includegraphics[width=12cm, height=5cm]{imag/circuito-RC.jpeg}
      \caption{\label{fig: circuito-RC} Circuito RC}
\end{figure}

\begin{figure}
      \centering
      \includegraphics[width=12cm, height=5cm]{imag/circuito-RL.jpeg}
      \caption{\label{fig: circuito-RL} Circuito RL}
\end{figure}

\begin{figure}
      \centering
      \includegraphics[width=12cm, height=5cm]{imag/circuito-RLC.jpeg}
      \caption{\label{fig: circuito-RLC} Circuito RLC}
\end{figure}




%=========Materiales===============
\section{Materiales}
\begin{itemize}
      \item 1 osciloscopio con 2 puntas de prueba,
      \item 1 generador de señal,
      \item 1 caja de resistencias décadas,
      \item 1 caja de condensadores décadas,
      \item 1 bobina de 600 vueltas,
      \item 1 transformador de $200/6 volt$,
      \item 6 conexiones.
\end{itemize}

%   \begin{itemize}

%   \end{itemize}

%================Procedimiento=================================
\section{Procedimiento Experimental}
% \subsection*{Parte 1: Medidas de voltaje y tiempo con osciloscopio}






















\subsection*{Parte 2: Circuitos RC, RL, RLC, en estado transitorio.}
\subsection*{\textbf{Circuito RC}}
\begin{enumerate}
      \item Procedemos a realizar el montaje del circuito experimental de la figura \ref{fig: circuito-RC} $(b)$, en el cual
      $R$ se asigna desde la caja resistencias décadas, por ej. $1 K\Omega$; $C$ se asigna desde la caja de condensadores décadas
      , por ej. $0.1 \mu F$ y $v_g$ corresponde a la señal de salida del generador, con forma rectangular y frecuencia apropiada, por
      ej. $1 K Hz$.

      \item Para observar $v_e(t)$ (voltaje sobre el condensador), se conecta la punta activada $CH1$ del osciloscopio al punto $c$, del circuito; se regulan los controles de base 
      de tiempo y niveles de ganancia, hasta observar en forma nítida la gráfica $v_e(t) vs t$. Para efectos de comparar y observar relación causa-efecto se utiliza el segundo canal 
      $(CH2)$ del osciloscopio conectado directamente a la salida del generador de señal y así tener a la vista el voltaje $v_g (t) vs t$ que se está aplicando al circuito.

      \item Mediante un manejo eficiente de los controles de barrido del osciloscopio y frecuencia del generador de señal se puede observar el efecto y significado de la constante de tiempo $\tau_C = RC$. Una vez hechos 
      los ajustes y optimizada la visualización, se hacen las medidas. 

      \item Para observar $v_R(t)$ o voltaje en la resistencia, que en realidad es una muestra de cómo varía la corriente $i(t)$ en la carga y en la descarga del condensador ya que $v_R(t) = i(t) R$, todo lo que hay que hacer
       es intercambiar las conexiones que van desde el circuito a la salida del generador de señal en $a$ y $b$, de esta manera, un extremo de $R$ quedará a tierra. Nos aseguramos que la punta activa de $CH2$ siga estando en la 
       salida a del generador y la de $CH1$ en $c$.
\end{enumerate}

\subsection*{\textbf{Circuito RL.}}

\begin{enumerate}
      \item Armamos el circuito de la figura \ref{fig: circuito-RL} $(b,)$ donde del condensador $C$ del circuito anterior se debe reemplazar por una bobina de $600$ vueltas y autoinducción $L = 9 mH$.

      \item Para efectos de control y buena visualización de $v_L(t)$ y $v_R(t)$, conviene ajustar $R$ en $300$ y usar frecuencia de $3 KHz$ en el generador de señal. La mejor visualización se logra reajustando
      el control de frecuencia.
      
      \item En lo que sigue, se repiten los pasos $2$, $3$ y $4$ de la parte anterior, con la diferencia de que lo observado es ahora $v_L (t)$, en vez de $v_C (t)$ en $2$, así como la constante de tiempo en $3$ 
      es ahora $\tau_L = L/R$.
\end{enumerate}


\subsection*{\textbf{Circuito RLC. Oscilaciones amortiguadas. Observación y medición.}}
\begin{enumerate}
      \item Se arma el circuito de la figura \ref{fig: circuito-RLC} $(b)$ con $R = 200\Omega$, $C = 0,01 \mu F$ y $L = 9 mH$ (bobina de $600$ vueltas). Utilizamos solamente $CH1$ conectada en $c$ para visualizar $v_C(t)$ donde se 
      observarán las oscilaciones. 

      % \item Proceder a cambiar los valores de $$R$ con fines de investigar su influencia en el decaimiento de la onda y en el número de oscilaciones (ondas) observadas, inclusive puede eliminarse por completo la resistencia de la caja de décadas.
      
      \item Medimos el período de las oscilaciones amortiguadas, $T'$, para obtener $w' = \frac{2 \pi}{T}$, esta  se comparará con el valor teórico:
      
      \begin{align*}
            w' = \left(w^2 - \left(\frac{R}{2L}\right)^2\right)^{1/2}
      \end{align*}
\end{enumerate}





%================Resultados===================================
Además para cada experimento hicimos los siguientes cálculos:\\
1 subdivision son 0.2 divisiones.\\
Amplitud del voltaje $(Am_v)$ = numero de diviones en el eje y (eje de voltaje).\\
Amplitud del período $(Am_T)$ = numero de divisiones en el eje x (eje del tiempo).\\ 
\begin{equation}\label{voltaje_max}
V_{max} = Am_v \times Volt/div 
\end{equation}

\begin{equation}\label{voltaje_eff}
V_{eff} = \dfrac{V_{max}}{\sqrt{2}}
\end{equation}

\begin{equation}\label{T_periodo}
T = Am_T \times Time/div 
\end{equation}

\begin{equation}\label{frecuencia}
f = \dfrac{1}{T}
\end{equation}

\begin{table}{h}
	\begin{center}
		\begin{tabular}{| c | c | c | c | c | c | c |}
			Circuito & $Am_v$ & $Am_T$ & $V_{max}$ & $V_{eff}$ & $T$ & $f$ \\ \hline
			RC & 0.8 div & 1.4 div & 4 Volt & 2.828 Volt & 2.8 $\times$ $10^{-3}$ seg & 357.14 Hz \\
			RL & 0.8 div & 4.2 div & 4 Volt & 2.828 Volt & 4.2 $\times$ $10^{-4}$ seg & 2380.8 Hz  \\
			RLC & 1.2 div & 4.2 div & 2.4 Volt & 1.69 Volt & 4.2 $\times$ $10^{-4}$ seg   & 2380.9 Hz   \\ \hline
		\end{tabular}
		\caption{Tabla de resultados, los cuales fueron calculados con los datos de la guía}
		\label{Tabla}
	\end{center}
\end{table} 


















%============Análisis==========================================================
\section{Análisis}
En general un osciloscopio muestra las señales de voltaje como ondas o representaciones visuales de la variación de voltaje vs el timepo, podemos ver que al hacer variar el voltaje, las señales mostradas en pantalla cambian sus características, como la amplitud y su período, en cambio, cuando variamos la frecuencia la señales mantienen su forma, pero cambian la cantidad de señales mostradas en pantalla.\\  
 
En el primer experimento conectamos el osciloscopio a un circuito RC, donde en la pantalla del osciloscopio podemos ver el voltaje en función del tiempo con una forma senoidal, a medida que aumentamos la resistencia y la capacitancia la amplitud de lads ondas decrece, más especificamente decrece exponencialmente a medida que aumenta el valor de $RC$, este resultado se asemeja a la función de voltaje para un circuito RC.Al elegir una capacitancia, resistencia, voltaje y frecuencia fijos, obtenemos los resultados de la tabla \ref{Tabla} para el circuito RC.\\

En el segundo experimento conectamos el osciloscopio a un circuito RL donde en la pantalla podemos ver que el voltaje vs tiempo tiene una forma de onda cuadrada, en la cual podemos ver que a medida que cambiamos el valor de la resitencia la forma de la señal aumenta o disminuye su amplitud, más especificamente la onda decae exponencialmente a medida que aumenta el valor de $R/L$.Donde para los valores entregados en la guía, los cuales son: autoinducción $L= 9mH$, $R = 300 \Omega$, y una frecuencia de $3 KHz$, obtenemos los resultados de la tabla \ref{Tabla} para el circuito RL.\\

En el tercer experimento conectamos el osciloscopio a un circuito RLC, donde en la pantalla podemos ver que  el voltaje en función del tiempo tiene una forma de onda senoidal pero amortiguada, en donde la inductancia del circuito se mantiene constante a $9mH$, y la capacitancia al hacer la variar del valor dado de $200 \Omega$ la señal no cambia en nada, pero a medida que hacemos variar la resistencia, cuando aumnetamos la resistencia la onda decae más rápidamente, en cambio cuando nos acercamos al punto de la resistencia a $0 \Omega$, las señales mostradas se vuelven extrañas, es como una superposición de muchas ondas, y con los valores dados en la guía de resistencia, inductancia y capacitancia de $0.01 \mu F$, obtenemos los resultados de la tabla \ref{Tabla} para el circuito  RLC.\\

Comparando los resultados de cada experimento podemos ver que el voltaje máximo de el circuito RC y RL es el mismo, es decir, la amplitud de las ondas es la misma, en cambio para el circuito RLC  es más pequeña que el de los otros dos,esto es debido a que como son ondas amortiguadas ...... , el voltaje efectivo más bajo es el del circuito RLC, por lo que el valor de la corriente alterna necesaria para generar la misma potencia que una corriente directa es menor, y por ende el circuito es más eficiente.\\





%============Conclusión========================================================
\section{Conclusión}
En este laboratorio pudimos conocer la forma de la gráfica de voltaje vs tiempo para tres distintas situaciones, el osciloscopio conectado a un circuito RC, RL y RLC, además de indentificar el cambio en la forma de las señales, 
de su voltaje máximo, efectivo, período y frecuencia, en donde logramos evidenciar que los experimentos comprobaban las ecuaciones que describen el voltaje en función del tiempo, del valor de la resistencia, del valor de la inductancia y del valor de la capacitancia  



\begin{thebibliography}{5}
  \bibitem{voltaje} Medidas de voltaje: Guía. (s. f.). NI. Recuperado 4 de noviembre de 2022, 
  de \url{https://www.ni.com/es-cl/support/documentation/supplemental/21/how-to-measure-voltage.html}
  \bibitem{Noe}Corriente eléctrica y materiales conductores. (2014, 5 septiembre). RedUSERS. 
  \url{https://www.redusers.com/noticias/corriente-electrica-y-materiales-conductores/}
  \bibitem{libro} \textbf{D. Halliday; R. Resnick; K. S. Kane.} \textit{Física Vol. 2.} (Cap.32), Compañía Editorial Continental, S.A. de C.V. 3º Edición, 1994
  \bibitem{benja} Circuito Eléctrico: Historia. (s. f.). Recuperado 4 de noviembre de 2022,
   de \url{https://www.profesorenlinea.cl/mediosocial/Circuito_ElectricoHistoria.htm}
\end{thebibliography}




\end{document}